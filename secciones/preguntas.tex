\begin{enumerate}
\item ¿Cuántos primos relativos hay en $Z_{256}$?


Primero, vamos a descomponer en factores primos a 256:

  $$ 256 = 2^8$$
  

Con la fórmula del totiente de Euler podemos saber el númerp de primos relativos:

 $$\phi(256) = 256 \times \frac{1}{2} = 128$$


Por lo tanto, el número de primos relativos en $ Z_{256} $ es $ \phi(256) = 128 $.

\item Sea, con A un alfabeto cualquiera. ¿qué se debe
cumplir para que f sea una función biyectiva?
\item ¿Cuántas posibles combinaciones no triviales existen para cifrar bytes con César, Decimado
y Afin?
\begin{itemize}
\item \textbf{Cifrado Cesar}

Primero debemos notar que los bytes tienen 256 valores.
Para el \textbf{Cifrado César},  sabemos funciona con un desplazamiento, en el que cada byte se reemplaza con $(x + k) mod 256$ donde k es la llave del desplazamiento. Por lo tanto, si se está usando un desplazamiento, solo tenemos tantas combinaciones como valores, por lo que tenemos \textbf{256 combinaciones}.

\item \textbf{Cifrado Decimado}
El cifrado decimado funciona cuando multiplicamos el valor del byte x por un llave k, para después hacer $mod 256$. Entonces, si qieremos invertir esta operación, $k$ debe der ser primo relativo  de 256, y como ya vimos anteriormente, 256 solo tiene 1\textbf{28 primos relativos} 

\item \textbf{Cifrado Afin}
El cifrado afín es una combinación del cifrado César y el cifrado Decimado, el cuál utiliza la siguiente función $(a\cdot x + b) mod 256$ en el que a es una constante primo relativo y b es una constante de desplazamiento.
Así que la combinación de los dos nos da $128 \cdot 255 = 32, 640$
\end{itemize}
\item ¿Por qué el sistema de archivos de UNIX, aunque un archivo tenga una extensión diferente
(o incluso no tenga), sigue reconociendo al archivo original?
\item ¿Por qué los archivos descifrados tienen exactamente el mismo tamaño que antes de cifrar,
pero no pudimos leerlos? ¿Por qué no tuvimos que agregar/quitar nada?
\item Ya que base64 no es un cifrado, sino codificación, ¿en qué casos podemos usarlo?
\item Supongamos que estuvieras en Hogwarts y tuvieras que utilizar un búho para comunicarte,
¿cuál crees que sería la mejor opción para mandar mensajes seguros a través de la lechuza?
\end{enumerate}